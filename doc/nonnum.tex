This is the documentation of the \texttt{MODULE NonNum}, a set
of \texttt{FORTRAN 90} routines to sort and search. This module make
use of the \texttt{MODULE NumTypes}, and \texttt{MODULE Error} so
please read the documentation of these modules \emph{before} reading
this. 

\section{Subroutine \texttt{Qsort(X,Ipt)}}
\index{Qsort@Subroutine \texttt{Qsort(X,Ipt)}}

\subsection{Description}

Sort the elements of \texttt{X(:)} in ascendant order.

\subsection{Arguments}

\begin{description}
\item[\texttt{X(:)}: ] Integer, real single or real double precision one
  dimensional array. Initially it contains unsorted numbers, and after
  calling the routine, it contains the sorted elements. \emph{Note that
    $X$ is overwritten when calling this routine}. 
\item[\texttt{Ipt(:)}: ] Integer vector, Optional. It returns the
  permutation made to \texttt{X(:)} to sort it.
\end{description}

\subsection{Examples}

\begin{lstlisting}[emph=Qsort,
                   emphstyle=\color{blue},
                   frame=trBL,
                   caption=Sorting data.,
                   label=qsort]
Program TestNN

  USE NumTypes
  USE NonNumeric

  Integer, Parameter :: Nmax = 10
  Integer :: Ima(Nmax)
  Real (kind=DP) :: X(Nmax), Y(Nmax)


  ! Fill X(:) with random data, and define Y(:)
  CALL Random_Number(X)
  Y = Sin(12.34_DP*(X-0.5_DP))

  ! Plot an unsorted data table
  Do I = 1, Nmax
     Write(*,'(1000ES13.5)')X(I), Y(I)
  End Do

  ! Sort them, and plot the table again. Same points, but this time
  ! sorted 
  CALL Qsort(X, Ima)
  Write(*,*)'# Again, this time sorted: '
  Do I = 1, Nmax
     Write(*,'(1000ES13.5)')X(I), Y(Ima(I))
  End Do


  Stop
End Program TestNN
\end{lstlisting}


\section{Function \texttt{Locate(X, $\mathtt{X_0}$, Iin)}}
\index{Locate@Function \texttt{Locate(X, $\mathtt{X_0}$, Iin)}}

\subsection{Description}

Given a \emph{sorted} vector of elements \texttt{X(:)}, and a point
$\mathtt{X_0}$, \texttt{Locate} returns the position $n$ such that
$\mathtt{X(n)<X_0<X(n+1)}$. If  $\mathtt{X_0}$ is less than all the
elements of $\mathtt{X(:)}$, \texttt{Locate} returns $0$, and if it is
greater than all the elements of $\mathtt{X(:)}$, it returns the
number of elements of $\mathtt{X(:)}$

\subsection{Arguments}

\begin{description}
\item[\texttt{X(:)}: ] Integer, real single or real double precision one
  dimensional \emph{sorted} array. 
\item[$\mathtt{X_0}$: ] Integer, real single or real double precision
  number, but the same type as $\mathtt{X(:)}$. Point that we want to
  locate in the sorted vector $\mathtt{X(:)}$.
\item[\texttt{Iin}: ] Integer, Optional. Initial guess of the position.
\end{description}

\subsection{Output}

Integer. The position $n$ such that 
\begin{displaymath}
  \mathtt{X(n)<X_0<X(n+1)}  
\end{displaymath}

\subsection{Examples}

\begin{lstlisting}[emph=Locate,
                   emphstyle=\color{blue},
                   frame=trBL,
                   caption=Searching data position in an ordered list.,
                   label=locate]
Program TestNN

  USE NumTypes
  USE NonNumeric

  Integer, Parameter :: Nmax = 100
  Integer :: Ima(Nmax), Idx
  Real (kind=DP) :: X(Nmax), Y(Nmax), X0


  ! Fill X(:) with random data, and set X0 to some arbitrary value. 
  CALL Random_Number(X)
  X0 = 0.276546754_DP

  ! Sort X(:), find the position of X0, and plot the neightborr
  ! elements. 
  CALL Qsort(X)
  Idx = Locate(X, X0)
  Write(*,'(1A,1ES33.25)')'Searched element: ', X0
  Write(*,'(1A,1ES33.25)')'Previous element in the list: ', X(Idx)
  Write(*,'(1A,1ES33.25)')'Next element in the list:     ', X(Idx+1)

  Stop
End Program TestNN
\end{lstlisting}

% Local Variables: 
% mode: latex
% TeX-master: "lib"
% End: 
