
This is the documentation of the MODULE \texttt{LapackAPI}, a set
of routines to perform calls to the Linear Algebra PACKage. 



\section{Subroutine \texttt{SVD(A, L [, U, VT])}}
\index{SVD@Subroutine \texttt{SVD(A [, U, VT]) }}

\subsection{Description}

The routine \texttt{SVD}, performs the singular value decomposition of
the matrix $A(:,:)$, optionally returning the right and left
eigenvectors. 

\subsection{Arguments}

\begin{description}
\item[\texttt{A}:]  Real double precision two dimensional array. The
  matrix whose singular value decomposition we want to know.

\item[\texttt{L(:): }] Real double precision one dimensional array. A
  vector containing the singular values of $A$.

\item[\texttt{U(:,:):}] Real double precision two dimensional
  array. Output. Optional. The left eigenvectors as columns.

\item[\texttt{VT(:,:):}] Real double precision two dimensional
  array. Output. Optional. The right eigenvectors as columns.
\end{description}

\subsection{Example}

\begin{lstlisting}[emph=SVD,
                   emphstyle=\color{blue},
                   frame=trBL,
                   caption=Using lapack library to perform a SVD decomposition.,
                   label=svd]
Program SVDtest

  USE NumTypes
  USE LapackAPI

  Real (kind=DP), Allocatable :: A(:,:), U(:,:), VT(:,:), Sv(:),&
       & SA(:,:)

  Real (kind=DP), Allocatable :: Work(:)
  Integer :: NWork
  Character (len=100) :: foo

  Write(*,*)'Enter matrix size: '
  Read(*,*)N
  M = N

  ALLOCATE(A(N,N), SA(N,N), U(N,N), VT(N,N), Sv(N))
  CALL Random_Number(A)
  A(:,N) = A(:,1)
  Write(*,*)'A = ['
  Do K1 = 1, N
     Write(foo,*)'(1X, ', N,'ES13.5,1A1)'
     Write(*,foo)(A(K1,K2), K2=1, N), ';'
  End Do
  Write(*,*)']'

  CALL SVD(A, Sv, U, VT)
  Write(*,*)Sv

  Write(*,*)'U = ['
  Do K1 = 1, N
     Write(foo,*)'(1X, ', N,'ES13.5,1A1)'
     Write(*,foo)(U(K1,K2), K2=1, N), ';'
  End Do
  Write(*,*)']'

  Write(*,*)'VT = ['
  Do K1 = 1, N
     Write(foo,*)'(1X, ', N,'ES13.5,1A1)'
     Write(*,foo)(VT(K1,K2), K2=1, N), ';'
  End Do
  Write(*,*)']'

  Stop
End Program SVDtest
\end{lstlisting}

\section{Function \texttt{PseudoInverse(A [, Ikeep])}}
\index{PseudoInverse@Function \texttt{PseudoInverse(A [, Ikeep])}}

\subsection{Description}

This function returns the pseudo-inverse of the matrix $A$.

\subsection{Arguments}

\begin{description}
\item[\texttt{A}:]  Real double precision two dimensional array. The
  matrix whose pseudo-inverse we want to know.

\item[\texttt{Ikeep: }] Integer. Optional. Number of singular values
  that we want to keep to perform the pseudo inversion. If not present
  the inverse matrix is computed. 

\end{description}

\subsection{Output}

A two dimensional array. Contains the pseudo-inverse of $A(:,:)$.

\subsection{Example}

\begin{lstlisting}[emph=PseudoInverse,
                   emphstyle=\color{blue},
                   frame=trBL,
                   caption=Using lapack library to compute the inverse.,
                   label=pseudoinverse]
Program SVDtest

  USE NumTypes
  USE LapackAPI

  Real (kind=DP), Allocatable :: A(:,:), B(:,:)
  Character (len=100) :: foo
  

  Write(*,*)'Enter matrix size: '
  Read(*,*)N

  ALLOCATE(A(N,N), B(N,N))
  CALL Random_Number(A)
  Write(*,*)'A = ['
  Do K1 = 1, N
     Write(foo,*)'(1X, ', N,'ES13.5,1A1)'
     Write(*,foo)(A(K1,K2), K2=1, N), ';'
  End Do
  Write(*,*)']'

  B = PseudoInverse(A)

  Write(*,*)'B = ['
  Do K1 = 1, N
     Write(foo,*)'(1X, ', N,'ES13.5,1A1)'
     Write(*,foo)(B(K1,K2), K2=1, N), ';'
  End Do
  Write(*,*)']'

  A = MatMul(A,B)
  Write(*,*)'A = ['
  Do K1 = 1, N
     Write(foo,*)'(1X, ', N,'ES13.5,1A1)'
     Write(*,foo)(A(K1,K2), K2=1, N), ';'
  End Do
  Write(*,*)']'

  Stop
End Program SVDtest
\end{lstlisting}





% Local Variables: 
% mode: latex
% TeX-master: "lib"
% End: 

