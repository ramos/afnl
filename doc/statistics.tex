This is the documentation of the \texttt{MODULE Statistics}, a set
of \texttt{FORTRAN 90} routines to perform statistical description
of data. This module make use of the \texttt{MODULE NumTypes},
\texttt{MODULE Constants}, \texttt{MODULE Error} and \texttt{MODULE
  Linear} so please read the documentation of these modules
\emph{before} reading this. 

\section{Function \texttt{Mean(X)}}
\index{Mean@Function \texttt{Mean(X)}}

\subsection{Description}

Compute the mean value of the numbers stored in \texttt{X(:)}.

\subsection{Arguments}

\begin{description}
\item[\texttt{X(:)}:] Double (DP) or simple (SP) precision one
  dimensional array. The values  whose mean we want to compute.
\end{description}

\subsection{Output}

A real double or simple precision (same type as the input). The mean
of the values.

\subsection{Examples}

\begin{lstlisting}[emph=Mean,
                   emphstyle=\color{blue},
                   frame=trBL,
                   caption=Computing the Mean of a vector of numbers.,
                   label=mean]
Program Tests

  USE NumTypes
  USE Error
  USE Statistics

  Integer, Parameter :: Nmax = 100, Npinta = 100, Npar = 4
  Real (kind=DP) :: X(Nmax), Y(Nmax), Yer(Nmax), &
       & Coef(Npar), Cerr(Npar), Corr, Xd(Nmax,2)


  CALL Random_Number(X)
  Write(*,'(ES33.25)')Mean(X)


  Stop
End Program Tests
\end{lstlisting}


\section{Function \texttt{Var(X)}}
\index{Var@Function \texttt{Var(X)}}

\subsection{Description}

Compute the variance of a vector of numbers \texttt{X(:)}

\subsection{Arguments}

\begin{description}
\item[\texttt{X(:)}:] Double (DP) or simple (SP) precision one
  dimensional array. The values  whose variance we want to compute.
\end{description}

\subsection{Output}

A real double or simple precision (same type as the input). The
variance of the values.

\subsection{Examples}

\begin{lstlisting}[emph=Var,
                   emphstyle=\color{blue},
                   frame=trBL,
                   caption=Computing the Variance of a set of numbers.,
                   label=var]
Program Tests

  USE NumTypes
  USE Error
  USE Statistics

  Integer, Parameter :: Nmax = 100, Npinta = 100, Npar = 4
  Real (kind=DP) :: X(Nmax), Y(Nmax), Yer(Nmax), &
       & Coef(Npar), Cerr(Npar), Corr, Xd(Nmax,2)


  CALL Random_Number(X)
  Write(*,'(ES33.25)')Var(X)


  Stop
End Program Tests
\end{lstlisting}


\section{Function \texttt{Stddev(X)}}
\index{Stddev@Function \texttt{Stddev(X)}}

\subsection{Description}

Computes the standard deviation of the numbers stored in the vector
\texttt{X(:)}. 

\subsection{Arguments}

\begin{description}
\item[\texttt{X(:)}:] Double (DP) or simple (SP) precision one
  dimensional array. The values  whose standard deviation we want to
  compute. 
\end{description}

\subsection{Output}

Real Single or Double precision, the same as the input values. The
standard deviation of the values.

\subsection{Examples}

\begin{lstlisting}[emph=Stddev,
                   emphstyle=\color{blue},
                   frame=trBL,
                   caption=Compputing the standard deviation.,
                   label=stddev]
Program Tests

  USE NumTypes
  USE Error
  USE Statistics

  Integer, Parameter :: Nmax = 100, Npinta = 100, Npar = 4
  Real (kind=DP) :: X(Nmax), Y(Nmax), Yer(Nmax), &
       & Coef(Npar), Cerr(Npar), Corr, Xd(Nmax,2)


  CALL Random_Number(X)
  Write(*,'(ES33.25)')Stddev(X)


  Stop
End Program Tests
\end{lstlisting}

\section{Function \texttt{Moment(X, k)}}
\index{Moment@Function \texttt{Moment(X, k)}}

\subsection{Description}

Returns the $k^{\underline{th}}$ moment of the values stored in the
vector \texttt{X(:)}.

\subsection{Arguments}

\begin{description}
\item[\texttt{X(:)}:] Real (Single or Double precision). The numbers
  whose $k^{\underline{th}}$ moment we want to compute.
\item[\texttt{k}:] Integer. Which moment we want to compute.
\end{description}

\subsection{Output}

Real single or double precision. The $k^{\underline{th}}$ moment of
the numbers.

\subsection{Examples}

\begin{lstlisting}[emph=Moment,
                   emphstyle=\color{blue},
                   frame=trBL,
                   caption=Computing the k$^{\text{\underline{th}}}$
                   moment of a data set.,
                   label=moment]
Program Tests

  USE NumTypes
  USE Error
  USE Statistics

  Integer, Parameter :: Nmax = 100, Npinta = 100, Npar = 4
  Real (kind=DP) :: X(Nmax), Y(Nmax), Yer(Nmax), &
       & Coef(Npar), Cerr(Npar), Corr, Xd(Nmax,2)


  CALL Random_Number(X)
  Write(*,*)'We should obtain the same numbers twice: '
  Write(*,'(ES33.25)')Moment(X,2), Var(X)

  Stop
End Program Test
\end{lstlisting}

\section{Subroutine \texttt{Normal(X, [Rm], [Rsig])}}
\index{Normal@Subroutine \texttt{Normal(X, [Rm], [Rsig])}}

\subsection{Description}

Fills \texttt{X(:)} with numbers from a normal distribution with mean
\texttt{Rm}, and standard deviation \texttt{Rsig}. The parameters
\texttt{Rm} and \texttt{Rsig} are optional. If they are not given the
mean will be 0, and the standard deviation 1.

\subsection{Arguments}

\begin{description}
\item[\texttt{X(:)}:] Real (Single or Double precision) one
  dimensional array. A vector that will be filled with numbers
  according to the normal distribution.
\item[\texttt{Rm}:] Real (Single or Double precision), Optional. The
  mean of the normal distribution. If not present the default value
  is 0.
\item[\texttt{Rsig}:] Real (Single or Double precision), Optional. The
  standard deviation of the normal distribution.  If not present the
  default value is 1. 
\end{description}

\subsection{Examples}

\begin{lstlisting}[emph=Normal,
                   emphstyle=\color{blue},
                   frame=trBL,
                   caption=Obtaining numbers with a normal distribution.,
                   label=normal]
Program Tests

  USE NumTypes
  USE Error
  USE Statistics

  Integer, Parameter :: Nmax = 100
  Real (kind=DP) :: X(Nmax)


  CALL Normal(X, 1.23_DP, 0.345_DP)
  ! Now compute the mean and standard deviation of the data
  Write(*,*)'We should obtain 1.23 and 0.345: '
  Write(*,'(ES33.25)')Mean(X), Stddev(X)


  Stop
End Program Tests
\end{lstlisting}


\section{Subroutine \texttt{Laplace(X, Rm, Rb)}}
\index{Laplace@Subroutine \texttt{Laplace(X, Rm, Rb)}}

\subsection{Description}

Fills \texttt{X(:)} with numbers from a Laplace distribution with mean
\texttt{Rm}, and variance $2\mathtt{Rb}^2$. 

\subsection{Arguments}

\begin{description}
\item[\texttt{X(:)}:] Real (Single or Double precision) one
  dimensional array. A vector that will be filled with numbers
  according to the normal distribution.
\item[\texttt{Rm}:] Real (Single or Double precision). The
  mean of the Laplace distribution.
\item[\texttt{Rb}:] Real (Single or Double precision). The
  width of the Laplace distribution (i.e. The variance is
  $2\mathtt{Rb}^2$).
\end{description}

\subsection{Examples}

\begin{lstlisting}[emph=Laplace,
                   emphstyle=\color{blue},
                   frame=trBL,
                   caption=Obtaining numbers with a Laplace distribution.,
                   label=laplace]
Program Tests

  USE NumTypes
  USE Error
  USE Statistics

  Integer, Parameter :: Nmax = 100
  Real (kind=DP) :: X(Nmax)


  CALL Laplace(X, 1.23_DP, 1.0_DP)
  ! Now compute the mean and standard deviation of the data
  Write(*,*)'We should obtain 1.23 and sqrt(2): '
  Write(*,'(ES33.25)')Mean(X), Stddev(X)


  Stop
End Program Tests
\end{lstlisting}

\section{Subroutine \texttt{Histogram(Val, Ndiv, Ntics, Vmin, Vmax,
    h)}} 
\index{Histogram@Subroutine \texttt{Histogram(Val, Ndiv, Ntics, Vmin, Vmax, h)}} 

\subsection{Description}

Given a set of points \texttt{Val(:)}, this routine makes
\texttt{Ndiv} divisions between the minimum and the greatest value of
\texttt{Val} (respectively returned in \texttt{Vmin} and
\texttt{Vmax}), each of size \texttt{h} (also returned), and returns
in the integer vector \texttt{Nticks(:)} the number of points that are
in each interval. 

\subsection{Arguments}

\begin{description}
\item[\texttt{Val(:)}:] Real (Single or Double precision) one
  dimensional array. The original values.
\item[\texttt{Ndiv}: ] Integer. The number of divisions.
\item[\texttt{Nticks}:] Integer one dimensional array. \texttt{Ndiv(I)}
  Tells how many points of \texttt{Val(:)} are between
  $\mathtt{Vmin+(I-1)h}$ and $\mathtt{Vmin+Ih}$.
\item[\texttt{Vmin, Vmax}:] Real (Single or Double precision). The
  minimum and maximum values of \texttt{Val}.
\item[\texttt{h}:] Real (Single or Double precision). After calling
  the routine has the step of the division.
\end{description}

\subsection{Examples}

\begin{lstlisting}[emph=Histogram,
                   emphstyle=\color{blue},
                   frame=trBL,
                   caption=Making Histograms.,
                   label=histogram]
Program Tests

  USE NumTypes
  USE Error
  USE Statistics

  Integer, Parameter :: Nmax = 500000, Npinta = 100, Npar = 4, Ndiv = 100
  Real (kind=DP) :: X(Nmax), Y(Nmax), Yer(Nmax), &
       & Coef(Npar), Cerr(Npar), Corr, Xd(Nmax,2), &
       & Xmin, Xmax, h, Xac
  Integer :: Ntics(Ndiv)

  CALL Normal(X, 1.23_DP, 0.345_DP)
  CALL Histogram(X, Ndiv, Ntics, Xmin, Xmax, h)
  
  Do I = 1, Ndiv
     Xac = Xmin + (I-1)*h
     Write(*,'(1ES33.25,1I)')Xac, Ntics(I)
  End Do

  Stop
End Program Tests
\end{lstlisting}

\section{Subroutine \texttt{LinearReg(X, Y, Yerr, [Func], Coef, Cerr, ChisqrV)}} 
\index{LinearReg@Subroutine \texttt{LinearReg(X, Y, Yerr, [Func], Coef, Cerr, ChisqrV)}} 

\subsection{Description}

Given a set of points \texttt{X(:)} and \texttt{Y(:)}, this routine
performs a linear fit to a set of functions defined by
\texttt{Func}. 
\begin{displaymath}
  Y = \sum_i a_i f_i(X)
\end{displaymath}
This routine also performs multi-dimensional fitting, in which case
the points are specified as \texttt{X(:,:)}, where the first argument
tells which point, and the second which variable.

\subsection{Arguments}

\begin{description}
\item[\texttt{X(:[,:])}:] Real single or double precision one
  dimensional array (for a one dimensional fit) or two dimensional
  array (for a multidimensional fit). The
  independent variables. For a multidimensional fit, the first argument
  tells which point, and the second which variable. So the size of the
  array should be \texttt{X(Npoints,Ndim)}.
\item[\texttt{Y(:)}: ] Real single or double precision one dimensional
  array. The dependent
  variable.
\item[\texttt{Yerr(:)}:] Real single or double precision one
  dimensional array. The errors
  of the points. If you don't have them, you should put all of hem to
  some non-zero value.
\item[\texttt{Func}:] Optional. This routine define the functions to
  fit. An interface like this should be provided
\begin{verbatim}
Interface
   Function Func(Xx, i)
         
     USE NumTypes

     Real (kind=SP), Intent (in) :: Xx
     Integer, Intent (in) :: i
     Real (kind=SP) :: Func

   End Function Func
End Interface
\end{verbatim}
if you want to perform a one dimensional fitting, and like this
\begin{verbatim}
Interface
   Function Func(Xx, i)
         
     USE NumTypes

     Real (kind=SP), Intent (in) :: Xx(:)
     Integer, Intent (in) :: i
     Real (kind=SP) :: Func

   End Function Func
End Interface
\end{verbatim}
if it is a multidimensional fitting. Since you are making a fitting
to a function of the type
\begin{displaymath}
  Y = \sum_i a_i f_i(X)
\end{displaymath}
the values $f_i(X)$ are given by this function as \texttt{Func(X,
  I)}. If the functions are not specified (i.e. you don't put this
argument), a fit to a polynomial is made (this only work for
one-dimensional fittings).
\item[\texttt{Coef(:)}: ] Real single or double precision one
  dimensional array. The
  parameters that you want to determine.
\item[\texttt{Cerr(:)}:] Real single or double precision one
  dimensional array. The errors
  in the parameters.
\item[\texttt{ChiSqr}: ] Real single or double precision. The $\chi^2$
  per degree of freedom of the fit.
\end{description}

\subsection{Examples}

\begin{lstlisting}[emph=LinearReg,
                   emphstyle=\color{blue},
                   frame=trBL,
                   caption=Doing linear regressions.,
                   label=linearreg]
Program Tests

  USE NumTypes
  USE Error
  USE Statistics

  Integer, Parameter :: Nmax = 200, Npinta = 100, Npar = 4, Ndiv = 100
  Real (kind=DP) :: X(Nmax), Y(Nmax), Yer(Nmax), &
       & Coef(Npar), Cerr(Npar), Corr, Xd(Nmax,2), &
       & Xmin, Xmax, h, Xac
  Integer :: Ntics(Ndiv)

  Interface
     Function Fd(Xx, i)
       
       USE NumTypes
       
       Real (kind=DP), Intent (in) :: Xx(:)
       Integer, Intent (in) :: i
       Real (kind=DP) :: Fd
       
     End Function FD
  End Interface


  CALL Random_Number(Xd)
  Xd(:,:) = 10.0_DP*(Xd(:,:) - 0.8_DP)

  CALL Normal(Yer, 0.0_DP, 1.0E-3_DP)
  Y(:) = 12.34_DP*Xd(:,1)*sin(Xd(:,2)) - 2.23_DP + &
       & 0.67_DP*Xd(:,1)**2*Xd(:,2) +  0.23_DP*Xd(:,1) + Yer(:) 


  CALL LinearReg(Xd, Y, Yer, Fd, Coef, Cerr, Corr)
  
  ! This should print the adjusted parameters, 
  ! that have values: 12.34, -2.23, 0.67, 0.23
  Do I = 1, Npar
     Write(*,'(2ES33.25)')Coef(I), Cerr(I)
  End Do

  ! This prints the ChiSqr, that should be very 
  ! close to 1.
  Write(*,'(1A,1ES33.25)')'ChiSqr of the Fit: ', Corr


  Stop
End Program Tests

! ************************************
! *
Function Fd(X, i)
! *
! ************************************

  USE NumTypes

  Real (kind=DP), Intent (in) :: X(:)
  Integer, Intent (in) :: i
  Real (kind=DP) :: Fd

  If (I==1) Then
     Fd = 1.0_DP
  Else If (I==2) Then
     Fd = X(1)*sin(X(2))
  Else If (I==3) Then
     Fd = X(1)**2*X(2)
  Else If (I==4) Then
     Fd = X(1)
  End If

  Return
End Function FD
\end{lstlisting}


% Local Variables: 
% mode: latex
% TeX-master: "lib"
% End: 

